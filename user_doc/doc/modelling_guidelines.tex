\chapter{modelling guidelines}
\label{modelling-guidelines}

\section{Modelling Introduction}
\label{cfu-modelling-intro}

This page lists remarks on creating a software architecture and design document in general
and it lists hints on getting along with the tool crystal\_facet\_uml.

As all tools, this program has its strengths and weaknesses.
This page helps in making use of the strenghts.

\section{crystal\_facet\_uml Hints}
\label{cfu-modelling-tool-hints}

\subsection{Tree Structure}

Diagrams are organized as a tree.

Start the root of the tree explaining the document scope and structure.

At the second level of the tree, list the main areas to be shown,
for example based on the arc42 template https://arc42.org/overview/ :

\begin{itemize}
\item Introduction and Goals
\item Constraints
\item Context and Scope (show the system boundary, what is outside, what use-cases exist)
\item Solution Strategy
\item Building Block View,
\item Runtime View,
\item Deployment View,
\item Crosscutting Concepts,
\item Architectural Decisions (show alternatives, give a rationale),
\item Quality Requirements
\item Risks and Technical Debt
\item Glossary (table showing Context, Term and Description)
\end{itemize}

\subsection{Focus}

Put only few elements into each diagram. This increases understandability of the main purpuse of the diagram.

Put further aspects of a topic into a separate diagram. Do not hesitate to copy an element from one diagram
to the next. This is what crystal\_facet\_uml is good at: it keeps the model in sync.

\section{General Hints on Architecture Documentation}
\label{architecture-hints}

\subsection{Problem vs. Solution}

Distinguish things that are
\begin{itemize}
\item given constraints (problem space),
\item decisions, chosen and rejected alternatives and
\item the designed solution
\end{itemize}

\subsection{Names}

Names of things are crucial: If the reader gets a wrong understanding by the name of an element, a hundred correct sentences
of describing text cannot set this straight again.

\subsection{Description}

Every design element needs a description, maybe a list of responsibilities: What shall this element do, what is it for?
Names alone cannot explain a system part.

\subsection{Precise sentences}

Be precise: Write in active form, e.g. The persistence component shall store and retrieve
binary data records indentified by string-based keys.

\subsection{Distinguish similar things}

Things that are similar but not the same shall be different entities when modelling.
E.g. The process in which an example application runs may
be different from the storage location and may be different from the software-component.
These are three things:
Example\_App\_Process (Type: Node), Example\_App\_ObjectFile (Type:Artifact) and Example\_App\_SWComponent (Type:Component).

